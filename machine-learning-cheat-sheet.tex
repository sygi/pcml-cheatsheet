\documentclass[10pt,a4paper,landscape]{article}
\usepackage{multicol}
\usepackage{calc}
\usepackage{ifthen}
\usepackage[landscape]{geometry}
\usepackage{amsmath,amsthm,amsfonts,amssymb}
\usepackage{color,graphicx}
\usepackage{hyperref}
\usepackage{listings}
\usepackage{underscore}
\usepackage{todonotes}

% Cheatsheet style
\input{style.tex}

% Shorthand for \bf
\providecommand{\bf}[1]{\ensuremath{\bf{#1}}}

\pdfinfo{
  /Title (Machine Learning Cheat Sheet)
  /Creator (TeX)
  /Producer (pdfTeX 1.40.0)
  /Author (Dennis Meier, Jakub Sygnowski)
  /Subject (Machine Learning cheatsheet)
  /Keywords (machinelearning, ml, bayes, regression, classification)
}

% -----------------------------------------------------------------------

\begin{document}
\title{Machine Learning Cheat Sheet}

\raggedright
\footnotesize
\sffamily
\begin{multicols*}{4}

% multicol parameters
% These lengths are set only within the two main columns
%\setlength{\columnseprule}{0.25pt}
\setlength{\premulticols}{1pt}
\setlength{\postmulticols}{1pt}
\setlength{\multicolsep}{1pt}
\setlength{\columnsep}{2pt}

\begin{center}
  \Large{\underline{Machine Learning Cheat Sheet}}
\end{center}

% ----------
\section{Regression}
  We have a set of $N$ training examples of dimensionality $D$, e.g:
  
  $x_n = \begin{bmatrix} x_{n1} \quad x_{n2} \quad ... \quad x_{nD} \end{bmatrix}^T $

  We often put the examples into one matrix with extra column of $1$s:

  $ \widetilde{X} = \begin{bmatrix}
    1 \quad x_{11} \quad x_{12} \quad \ldots \quad x_{1D} \\
    \ldots\\
    1 \quad x_{N1} \quad x_{N2} \quad \ldots \quad x_{ND}
  \end{bmatrix}
  $

  (note that $x_n$ are rows, not columns)

  The goal is to predict a $\hat{y}$ given $x$.

  Simple linear regression: $y_n \approx \beta_0 + \beta_1 x_{n1}$

  Multiple dimension linear regression: $y_n \approx f(x^*) := \beta_0 + \beta_1 x^*_{1} + \beta_2 x^*_{2} + ... + \beta_D x^*_{D}$

  \subsection{Linear basis function model}
  $y_n = \beta_0 + \sum_{i=1}^{M} \beta_i \phi_i(\bf{x_n}) =  \bf{\widetilde\phi^T}(\bf{x}^T_n) \boldsymbol\beta$.
  The optimal $\beta$ is given by $\beta = ( \widetilde{\Phi}^T \widetilde{\Phi})^{-1} \widetilde{\Phi}^T y$ where $\widetilde{\Phi}$ is a matrix with N rows and the n-th row is $[1, \phi_1(x_n)^T,  ...,  \phi_M(x_n)^T]$.

  Ridge regression: $\beta_{ridge} = ( \widetilde{\boldsymbol \Phi}^T \widetilde{\boldsymbol\Phi} + \lambda \boldsymbol I)^{-1} \widetilde{\boldsymbol\Phi}^T \boldsymbol y$

  \subsection{Cost functions}
  \begin{colfig}
    \centering
    \includegraphics[width=\linewidth]{images/error-functions.png}
  \end{colfig}

  Cost function / Loss: $\mathcal{L}(\boldsymbol\beta) = \mathcal{L}(\mathcal{D},\boldsymbol\beta)$

  Mean square error (MSE): $\frac{1}{2N} \sum_{n=1}^{N}\left[y_n-f(\bf{x}_i) \right]^2$

  Mean absolute error (MAE): $\frac{1}{2N} \sum_{n=1}^{N}\left | y_n-f(\bf{x}_i) \right |$

  Huber loss: $\mathcal{L}_\delta (a) = \begin{cases}
   \frac{1}{2}{a^2}                   & \text{for } |a| \le \delta, \\
   \delta (|a| - \frac{1}{2}\delta ), & \text{otherwise.}
  \end{cases}$

  Root mean square error (RMSE): $\sqrt{2 * \text{MSE}}$

  Epsilon insensitive (used for SVMs):
  $\mathcal{L}_{\epsilon}(y, \hat{y}) = \begin{cases}
   0                   & \text{if } |y - \hat y| \le \epsilon, \\
   |y - \hat y| - \epsilon, & \text{otherwise.}
  \end{cases}$

% ----------

\todo[inline]{TODO: statistical/computational tradeoff}

% ----------
\section{Gradient Descent}
  General rule: $\boldsymbol\beta^{(k+1)} = \boldsymbol\beta^{(k)} - \alpha \frac{\partial \mathcal{L}(\boldsymbol\beta^{(k)})}{\partial \boldsymbol\beta}$

  How to get a good $\alpha$ is a hard question (too small gives slow time, too big may not converge).

  The gradient for MSE comes out as:
  $\frac{\partial \mathcal{L}}{\partial \boldsymbol\beta} = - \frac{1}{N} \widetilde{X}^T ( \boldsymbol y - \widetilde{X} \boldsymbol\beta )$

% ----------
\section{Normal equations}
  $\beta = ( \widetilde{X}^T \widetilde{X} )^{-1} \widetilde{X}^T y$

  Works for $\widetilde{X}$ having $rank = D$ (so that inverse is defined)
% ----------
\section{Classification}
  Logistic Function $\sigma = \frac{exp(x)}{1+exp(x)}$

  Classification with linear regression: Use $y = 0$ as class $\mathcal{C_1}$
  and $y = 1$ as class $\mathcal{C_2}$ and then decide a newly estimated $y$ belongs
  to $\mathcal{C_1}$ if $y < 0.5$.

  \subsection{Logistic Regression}
  $\widetilde{\bf{X}}^T [\sigma(\widetilde{\bf{X}} \beta) - y] = 0$
  \todo[inline]{TODO: Generalized Linear model}

  \subsection{Cost functions}

  Root Mean square error (RMSE): $\sqrt{\frac{1}{N} \sum_{n=1}^{N}\left[y_n- \hat{p_n} \right]^2}$

  0-1 Loss: $ \frac{1}{N} \sum_{n=1}^{N} \delta(y_n, \hat{y_n})$

  logLoss: $- \frac{1}{N}  \sum_{n=1}^{N} y_n \log(\hat{p_n}) + (1-y_n) \log(1-\hat{p_n})$

% ----------
\section{Occam's Razor}
  It states that among competing hypotheses, the one with the fewest assumptions should be selected. Other, more complicated solutions may ultimately prove correct, but—in the absence of certainty—the fewer assumptions that are made, the better.
% ----------
\section{Math}
convexity: 

$\forall_{x_1, x_2} \forall_{t\in[0,1]} f(tx_1 + (1-t)x_2) \le tf(x_1) + (1-t)f(x_2)$

Jensen's inequality (log is \textbf{concave}): $log(\frac{\sum_{i=1}^n x_i}{n}) \ge \frac{\sum_{i=1}^n log(x_i)}{n}$

Hessian is positive semidefinite $\Rightarrow$ function is convex.

Positive semidefinite matrix $M \Leftrightarrow$ Gram matrix (inner product) $\Leftrightarrow \forall_x x^T M x \ge 0$.

\subsection{Difficult words}
\textbf{consistent} estimator converges to the true value as we increase the
number of data to infinity.

\textbf{unidentifiable} model has many global minima due to symmetry.

\subsection{Distributions}
  Gaussian: $\mathcal{N}(X| \mu, \sigma^2)$ \\
  $\implies p(X = x) = \frac{1}{\sqrt{2 \pi \sigma^2}} \exp{(- \frac{1}{2} ( \frac{x - \mu}{\sigma} )^2)}$

  Poisson: $\mathcal{P}(X| \lambda)$ \\
  $\implies p(X = k) = \frac{\lambda ^ k}{k!} \exp{(- \lambda)}$

% ----------
\section{Complexities}
Grid search: $\mathcal{O}(M^D N D)$, where $M$ is the number of test points in one dimension.

Gradient descent: $\mathcal{O}(I N D)$ where $I$ is the number of iterations we make.

Least-squares (normal equations): $\mathcal{O}(ND^2 + D^3)$

Newton's method (with Hessian): $\mathcal{O}(ND^2 + D^3)$

% ----------
\section{Bias-Variance Decomposition}
  \begin{colfig}
    \centering
    \includegraphics[width=\linewidth]{images/bias-variance.png}
  \end{colfig}

  \begin{tabular}{ l || c | c }
                            & bias & variance \\
    \hline
    regularization          & +    & - \\
    reduce model complexity & +    & - \\
    more data               & -    & - \\
    \hline
  \end{tabular}

% ----------
\section{Maximum Likelihood}
  The Likelihood Function maps the model parameters to the probability distribution of $\bf{y}$:
  $\mathcal{L}_{lik}\colon \text{parameter space} \to [0;1]\quad  \bf{\beta} \mapsto p(\bf{y} \mid  \bf{\beta})$
  An underlying $p$ is assumed before. If the observed $y$ are IID, $p(\bf{y} \mid \beta) = \prod_n p(y_n \mid \beta)$.

  $\mathcal{L}_{lik}$ can be viewed as just another cost function. Maximum likelihood then simply choses the parameters $\bf{\beta}$ such that observed data is most likely. $\beta = \argmax_{\text{all} \beta} L(\beta)$

  Assuming different $p$ is basically what makes this so flexible. We can chose e.g.:

  \begin{tabular}{ l  l }
    \hline
    Gaussian $p$ & $\mathcal{L}_{lik} \widehat{=} \mathcal{L}_{MSE}$ \\
    Poisson $p$  & $\mathcal{L}_{lik} \widehat{=} \mathcal{L}_{MAE}$ \\
    \hline
  \end{tabular}

% ----------
\section{Bayesian methods}
  Bayes rule: $p(A, B) = p(A|B) p(B) = p(B|A) p(A)$

  The \textbf{prior} $p(\bf{f}|\bf{X})$ encodes our prior belief about the ``true'' model $\bf{f}$. The \textbf{likelihood} $p(\bf{y}|\bf{f})$ measures the probability of our (possibly noisy) observations given the prior.

  Least-squares tries to find model parameters $\bf{\beta}$ which maximize the likelihood. Ridge regression maximizes the \textbf{posterior} $p(\bf{\beta}|\bf{y})$

  % ----------
  \subsection{Graphical Models}
  \todo[inline]{TODO: Bayes Net: Directed acyclic graph}
  \todo[inline]{TODO: Belief propagation}
  Graph between the observations and the variables is a bi-partite graph.

% ----------
\section{Kernel}
  Basically, Kernels are a mean to measure distance, or ``similarity'' of two vectors. We define:

  $(\bf{K})_{i,j} = \kappa(\bf{x_i}, \bf{x_j}) = \vec \phi(\bf{x_i})^T \vec \phi(\bf{x_j})$.

  The $\phi$ are not that important in the end, because we only use the Kernel as is. Sometimes it's even impossible to write them down explicitly.

  \begin{tabular}{ l | l }
    \hline
    Linear     & $\kappa(\bf{x_i}, \bf{x_j}) = \bf{x_i}^T \bf{x_j}$ \\
    \hline
    Polynomial & $\kappa(\bf{x_i}, \bf{x_j}) = (\bf{x_i}^T \bf{x_j} + c)^d$ \\
    \hline
    RBF        & $\kappa(\bf{x_i}, \bf{x_j}) = \exp\left(-\frac{||\bf{x_i} - \bf{x_j}||^2}{2\sigma^2}\right)$ \\
    \hline
  \end{tabular}

% ----------
\section{Neural Networks}
  \todo[inline]{TODO: Intuition and notation, backpropagation, regularization techniques}

% ----------
\section{Support Vector Machines}
  Search for the hyperplane separating the data such that the gap is biggest.
  It minimizes the following cost function:

  $\mathcal{L}_{SVM} (\bf{\beta})= \sum_{n=1}^N [1 - y_n \widetilde\phi_n \beta]_{+} + \frac{\lambda}{2} \sum_{j=1}^M \beta_j^2$

  This is convex but not differentiable.

% ---------- Footer
\hrule
\tiny
Rendered \today. Written by Dennis Meier, Jakub Sygnowski.
\copyright Dennis Meier. This work is licensed under the Creative Commons Attribution-ShareAlike 3.0 Unported License.
To view a copy of this license, visit http://creativecommons.org/licenses/by-sa/3.0/ or
send a letter to Creative Commons, 444 Castro Street, Suite 900, Mountain View, California, 94041, USA.
\includegraphics{images/by-sa.png}

\end{multicols*}
\end{document}
